<p>Como vimos, a condição de tração exige o conhecimento das cargas sobre os eixos \(W_f\) e \(W_r\).<p>

TMA do reboque em relação ao eixo

\begin{equation} \label{eq:reboqueTMA}
	-W_b \cos \theta f + W_b \sin \theta \, h_3 - F_{xb} \, h_2 + F_{zb}(f+e) = 0
\end{equation}

TMB do reboque na direção X

\begin{equation} \label{eq:reboqueTMB}
	F_{xb} - W_b \sin \theta = 0
\end{equation}

TMA do carro em relação ao eixo dianteiro

\begin{equation} \label{eq:carroTMAf}
	W \cos \theta \, b + W \sin \theta \, h_1 - W_r (b+c) + F_{zb}(b+c+d) + F_{xb} h_2 = 0
\end{equation}

TMA do carro em relação ao eixo traseiro

\begin{equation} \label{eq:carroTMAr}
	-W \cos \theta \, c + W \sin \theta \, h_1 + W_f (b+c) + F_{zb} d + F_{xb} h_2 = 0
\end{equation}

